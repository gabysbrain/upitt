\documentclass[10pt]{article}

\usepackage{fullpage}
\usepackage{longtable}
\usepackage[pdftex]{graphicx}
\DeclareGraphicsExtensions{.pdf,.jpg}

% no numbers, etc
\pagestyle{empty}

\begin{document}

{\bf Research Methods Homework} \hfill {\raggedleft Thomas Torsney-Weir}

\section{Systems}
\subsection{MIT}
This system is limited to describing the weather in cities and telling me what 
it's name is.  When I say something unexpected it seems to try to process my
statement as one of the cities it knows about.  For example, when I told it
``you're awesome'' it responded with the weather for Boston.  When I asked it
``where are you?'' it responded with ``my name is Jupiter.''  You also had
to take care not to start talking before it finished talking and then beeped.
Otherwise, it would not understand what you said.

\subsection{Amtrak}
This system is not very helpful.  It required my responses to menu choices to
match the options it gave.  It also had trouble when I said I was leaving from
Pittsburgh.  It thought I said ``Denver, Colorado.''  I'm not sure how this
happened.  When I told it the number of passengers was ``43 children'' it
thought I meant 4 adults and 3 children.  I got frustrated with this system
pretty quickly.  I'd rather use their website for booking trains.

\section{Transcript}
Teacher: Now this law that force is equal to mass times accelleration, what's 
this law called? This is essentially a very important basic fact.  Uh, it is 
it is a law of physics.  Um, you have you have read it in the background 
material, can you recall it? \\
Student (Unsure): Ummm, huh no.  It was one of newton's laws but \\
Teacher: Right \\
Student (nervous): I don't remember which one \\
Teacher: That is newton's second law of motion \\
Student: Ok, cause I remember one, two, and three but I didn't know if there 
was a different name. \\
Teacher: Yeah, that's right.  You know newton was a genius.

\subsection{Issues}
Sometimes it's difficult to figure out exactly what is being said.  For 
example, I wasn't sure if the teacher said ``forces'' or ``force is'' in 
the first statement.  There's also the issue of how to annotate when two 
people are speaking together.  I suppose in dialog systems this is ok since
one person will usually stop to let the other speak but if you were trying
to transcribe all the discussions in a crowd things would be more difficult.
You might need to include some sort of time component in each annotation.
So for each line in the dialog you'd give the time period that person was
speaking.  When two people speak at the same time then they would have 
overlapping times.

\end{document}

