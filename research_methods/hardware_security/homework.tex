\documentclass[12pt]{article}

\usepackage{fullpage}
\usepackage{longtable}
\usepackage[pdftex]{graphicx}
\DeclareGraphicsExtensions{.pdf,.jpg}

% no numbers, etc
\pagestyle{empty}

\begin{document}

{\bf Research Methods Homework} \hfill {\raggedleft Thomas Torsney-Weir}

{\em An Integrated Framework for Dependable and Revivable Architectures 
Using Multicore Processors} introduces the INDRA system which is designed
to provide a system that is tolerant against remote attacks.  It focuses on
being able to monitor and recover against attacks.

INDRA uses a multicore architecture to accomplish its goal.  It splits the
cores into an execution class and a monitor/recovery class.  The monitor cores
run in an elevated security mode and even  a completely separate OS.  The
execution cores cannot access the memory space of the monitor class and the
monitor class cannot run network services thus is not vulnerable to a remote
attack.  

The monitoring cores watch execution on the other processors for events
that would signify an attack.  The monitoring cores can watch execution
in real time since they are located on the same die as the executing cores.
The monitors also take periodic memory images as checkpoints.  They use a
delta scheme to speed the checkpointing process.  When a fault is detected
the monitor can pause execution and restore from a checkpoint.

This paper provides a novel approach to dealing with remote attacks.  I
really like the fast checkpointing idea and using a second core to monitor
execution.  However, I'm not entirely sure this system would allow a system
to recover from a DoS attack.  It seems that one could launch a DoS attack
on the machine, wait for it to recover and then launch another attack.  
Depending on the architecture you could even DoS the monitoring core by
causing so many faults it spends all its time recovering from the attacks.

INDRA would also require a lot of memory to perform the checkpointing.  I
imagine this system would primarily be used in routers with specialized
hardware so maybe the memory used by the execution cores would not be that
large.  Even so, you would need double the memory just to do checkpointing.

Future work could extend INDRA to work on more common hardware.  This system
requires a special BIOS and it seems to require special registers and I/O
systems to handle the interactions between the cores.  

\end{document}

