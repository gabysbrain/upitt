\documentclass{article}

\usepackage{verbatim}

\author{Thomas Torsney-Weir}
\title{Is Video Poker Fair?}

%page formatting
\setlength{\parskip}{2.5ex plus 0.5ex minus 0.5ex}
\setlength{\parindent}{0ex}

\begin{document}

\maketitle

\section{Introduction}
Video poker works by having the player pay an initial fee.  The machine then
randomly picks five cards out of a standard 52 card deck.  The player is then 
offered the opportunity to exchange up to five cards for new ones from the
remaining deck.  Finally, the hand is evaluated and you receive a payoff 
--- which may be zero --- based on the quality of your hand.

I intend to investigate whether the payoffs for various video poker hands 
is fair.  In other words does the expected payoff for the various hands 
make it profitable for the player to play.

\section{Experiment}
I will simplify the game by removing the ``draw'' step.  This is the step
where the player is offered the chance to exchange some or all of their cards
for new ones.  I will only evaluate the original five card hand.  There is a
certain strategy in picking the best cards to exchange.  

\begin{comment}
longer explanation
You need to assume some prior probability for each class of hands given your 
current hand.  This needs to be combined with the probability of each card 
in the remaining deck, etc.
\end{comment}

This will make the programming simpler and remove any assumptions of strategy
from the hands generated in the experiment.

I will randomly generate five distinct cards per trial.  Each of the cards 
has an equal chance of occuring.  The hand will be evaluated to determine 
what class it is in.  The class determines the payoff.
Here are the classes listed in order from best to worst: royal flush, 
straight flush, four of a kind, full house, flush, straight, three of a 
kind, two pair, and one pair.  The payoff schedule for a one dollar fee in a
typical game is as follows: 250, 50, 25, 9, 6, 4, 3, 2, 1 for royal flush to
one pair respectively.  In the case where a hand can take multiple 
classes you use the best hand.  The value of each hand will be equal to 
the payoff.  I will also count the occurences of each class of hands.

The number of distinct poker hands is 5 choose 2.  There's no reason to 
generate any more than this number of hands.  However, these hands will 
be generated with a uniform probability.  The payoffs for the game follow
an exponential distribution.  I might need to generate more than the 5 choose
2 hands before I get a royal flush.  

\section{Analysis}
The average payoff is an interesting measure.  This is the amount 
you'd expect to receive when playing the game.  Hopefully this will be 
greater than the bet amount otherwise is uneconomical to play.  

The confidence interval is also a good measure as the payoff for 
a straight flush is so much greater than for two pair.  The average may
be skewed.  The confidence interval will tell you the range where your 
expected return lies.  The issue is that the player doesn't have enough 
money or time to play the game forever.  If the confidence interval is wide
then the player may never get the expected return.

The payoffs listed for the poker hands seem to follow an exponential 
distribution.  I will compare the graphs of the payoff function with 
a histogram of the hands I generated.  The shape of these graphs should
be similar if the payoffs for the various hands are fair.

%I will use the chi-square test to check whether the hypothesis that 

\end{document}

