\documentclass{article}

\usepackage{verbatim}

\author{Thomas Torsney-Weir}
\title{Is Video Poker Fair?}

%page formatting
\setlength{\parskip}{2.5ex plus 0.5ex minus 0.5ex}
\setlength{\parindent}{0ex}

\begin{document}

\maketitle

\section{Introduction}
Video poker works by having the player pay an initial fee.  The machine then
randomly picks five cards out of a standard 52 card deck.  The player is then 
offered the opportunity to exchange up to five cards for new ones from the
remaining deck.  Finally, the hand is evaluated and you receive a payoff 
--- which may be zero --- based on the quality of your hand.

I intend to investigate whether the payoffs for various video poker hands 
is fair.  In other words does the expected payoff for the various hands 
make it profitable for the player to play.

\section{Experiment}
I will simplify the game by removing the ``draw'' step.  This is the step
where the player is offered the chance to exchange some or all of their cards
for new ones.  I will only evaluate the original five card hand.  There is a
certain strategy in picking the best cards to exchange.  This will make the 
programming simpler and remove any assumptions of strategy from the hands 
generated in the experiment.

I will randomly generate five distinct cards per trial.  Each of the cards 
has an equal chance of occurring but can only occur once per hand.  The hand 
will be evaluated to determine what class it is in.  The class determines 
the payoff.  Here are the classes listed in order from best to worst: royal 
flush, straight flush, four of a kind, full house, flush, straight, three of 
a kind, two pair, and one pair.  The payoff schedule for a one dollar fee in a
typical game is as follows: 250, 50, 25, 9, 6, 4, 3, 2, 1 for royal flush to
one pair respectively.  In the case where a hand can take multiple 
classes you use the best hand.  The value of each hand will be equal to 
the payoff.  I will also count the occurrences of each class of hands.

The number of distinct poker hands is 5 choose 2.  There's no reason to 
generate any more than this number of hands.  However, these hands will 
be generated with a uniform probability.  The payoffs for the game follow
an exponential distribution.  I might need to generate more than the 5 choose
2 hands before I get a royal flush.  

\section{Results}
The first step was to test for convergence.  I ran a minimum of five sets of
100,000 hands each.  I then did a pair-wise T-test to see the chance my
distributions came from the same distribution.  The matrix of p values is below:
\begin{tabular}{|c||c|c|c|c|c|}
\hline 
   & x1   & x2   & x3   & x4   & x5   \\
\hline 
\hline 
x1 & 1    & 0.31 & 0.77 & 0.84 & 0.42 \\
\hline 
x2 & 0.31 & 1    & 0.41 & 0.22 & 0.82 \\
\hline 
x3 & 0.77 & 0.41 & 1    & 0.60 & 0.55 \\
\hline 
x4 & 0.84 & 0.22 & 0.60 & 1    & 0.30 \\
\hline 
x5 & 0.42 & 0.82 & 0.55 & 0.30 & 1    \\
\hline 
\end{tabular}

These numbers are all well above 0.05 so with over 95 percent confidence they
come from the same distribution.  There was no reason to run any more samples.
For the remainder of my experiment I combined these five experiments.

The mean of the payoffs is the expected return from playing the game.  In this
case the mean is 0.63 which means that you can expect a return of 63 cents on
the dollar when you play the game.  This would indicate that the game is not
fair.

To further evaluate the payoffs I did a Chi-squared test on the counts of
each class of hands against the probabilities of each hand type implied by 
the payoffs.  I assumed that the zero valued hand had a probability of 1
and the rest had decreasing probabilities in the following fashion: 1, 0.5,
0.33, 0.25, 0.2, 0.143, 0.1, 0.0385, 0.0196, 0.00398.  These were normalized
for the Chi-squared test.  The results of this test was a probability of
less than $2.2*10^{-16}$ so these distributions are vastly different.  The
$\chi^2$ value is 298972.5 and the number of degrees of freedom is 9.

\section{Conclusion}
In summary, the payoff from video poker does not seem worth it.  A player
could expect to lose almost 40 cents every time they play.  Furthermore, the
payoffs don't match the distribution of hands.  It might not be a good idea
to try and make a living from playing video poker.  On the other hand, there
is a fun component to the whole thing and it's difficult to gauge that.

\end{document}

