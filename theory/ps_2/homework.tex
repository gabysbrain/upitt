\documentclass[10pt]{article}

\nofiles

\usepackage{fullpage}
\usepackage[pdftex]{graphicx}
\DeclareGraphicsExtensions{.pdf,.jpg}

% no numbers, etc
\pagestyle{empty}

\begin{document}

{\bf Problem Set 2} \hfill {\raggedleft Thomas Torsney-Weir}

%Each homework problem goes here
\begin{enumerate}
\item % non-halting function
  \begin{verbatim}
INPUT(R_1)
N_1 <- 0
UNTIL Y_1 TRUE DO
  X_1 <- P(N_1)
  FOR X_1 TIMES DO
    Y_1 <- Y_1 + 1
  ENDFOR
ENDUNTIL
OUTPUT(Y_1)
  \end{verbatim}

\item 
  I have no idea what the inverse of a permutation is

\item % function f(x) = k is primitive recursive
  Base case: Let $k=0$. Then $f(x) = 0$ which is one of the base primitive 
  recursive functions.

  Hypothesis: $f(x) = k$ is primitive recursive.  This is equivalent 
              to $f(k, x) = k$.

  Proof:  We can build $f(k+1,x) = k+1$ by using the sucessor function.
  \begin{eqnarray*}
    f(k+1, x) & = & k + 1 \\
    f(k+1, x) & = & f(k, x) + 1 \\
    f(k+1, x) & = & S(P_2^3(k, f(k, x), x)) 
  \end{eqnarray*}

\item % f(x) = h(x) + g(x)
  From the notes $add(y, x)$ is primitive recursive so
  \begin{eqnarray*}
    h(x) & = & f(x) + g(x) \\
         & = & add(P_1^2(f(x), g(x)), P_2^2(f(x), g(x)))
  \end{eqnarray*}

\item % nth iteration
  Base case: $\iota(0, x) = x$.  This is primitive recursive by assumption.
  
  Hypothesis: $\iota(n, f) = f^n(x)$ is primitive recursive.

  Proof: 
  \begin{eqnarray*}
    \iota(n+1, f) & = & f(\iota(n, f)) \\
    \iota(n+1, f) & = & f(P_2^3(n, \iota(n, f), f))
  \end{eqnarray*}

\item % exam question
  Assume that $f(n, x)$ is primitive recursive.  Let $h(n, x) = f(n, f(n, x))$.
  This function is primitive recursive since
  \begin{eqnarray*}
    h(n, x) & = & f(n, f(n, x)) \\
    h(n, x) & = & f(P_1^2(n, f(n, x)), P_2^2(n, f(n, x)))
  \end{eqnarray*}
  and
  \begin{eqnarray*}
    f(n+1, x) & = & f(n, f(n, x)) \\
    f(n+1, x) & = & h(n, x) \\
    f(n+1, x) & = & P_2^3(n, h(n, x), x)
  \end{eqnarray*}

\item % perfect square
  Let 
  \begin{math}
    P(m, x) = \left\{ \begin{array}{ll}
      1 & if m * m = x \\
      0 & otherwise
        \end{array} \right.
  \end{math} be a predicate. Also let $y(x) = \min m \le x[P(m, x)]$ be a
  bounded minimization problem.  So, $f(x) = add(y(x), y(x))$.

\item % gcd
  ???

\item % lcm
  Let $div(x, y) = \frac{x}{y}$ be the integer division function.  This is
  primitive recursive because it can be defined as a bounded minimization
  problem.  Also let $mult(x, y) = x * y$ be the multiplication function.
  \begin{eqnarray*}
    lcm(x, y) & = & \frac{xy}{gcd(x, y)} \\
    lcm(x, y) & = & div(mult(x, y), gcd(x, y)) 
  \end{eqnarray*}
  
  Since $gcd(x, y)$ is primitive recursive, $lcm(x, y)$ is primitive recursive.

\item % Theorem 4.5 part 2
  Induction hypothesis: 
    \begin{eqnarray*}
      g(\vec{x}_2^n)         & = & \nu(f_P(\kappa(x_2,\ldots,\kappa(x_n)))) \\
      h(y+1, z, \vec{x_2^n}) & = & \nu(f_Q(\kappa(y), \kappa(z), 
                                           \kappa(x_2),\ldots,\kappa(x_n)))) \\
    \end{eqnarray*}
  So,
    \begin{eqnarray*}
      \nu(Z_1) & = & \nu(f_Q(\kappa(y), \kappa(z), 
                             \kappa(x_2,\ldots,\kappa(x_n)))) \\
               & = & \nu(f_Q(\kappa(y), f_Q(\kappa(y), f_Q(\kappa(y-1), 
                                               \ldots, 
                                               \kappa(x_2),\ldots,\kappa(x_n)), 
                             \kappa(x_2),\ldots,\kappa(x_n))))) \\
               & = & \nu(f_Q(\kappa\circ y, h(y, z, \vec{x_2^n}), 
                                         x_2,\ldots,x_n)) \\
               & = & \nu \circ \kappa \circ h(y+1, h(y, z, \vec{x_2^n}), 
                                              \vec{x_2^n}) \\
               & = & h(y+1, h(y, z, \vec{x_2^n}), \vec{x_2^n})
    \end{eqnarray*}

\item % exercise 4.1
  \begin{enumerate}
    \item 
      Let $h(y) = m \le y[P(m, \vec{z}^n)]$.  So
      $f_1(x) = \min y \le g(x)[P(y, \vec{z}^n)] = h(g(x))$.
    \item ???
    \item ???
    \item
      $\chi(f_1(x))$
  \end{enumerate}

\item % exercise 4.2
  \begin{enumerate}
    \item $f_1(x, y) = not(x - y)$.
    \item $f_2(x, y) = f_1(x, y) - f_2(y, x)$
    \item $f_3(x, y) = not(f_2(x, y))$
    \item 
      \begin{eqnarray*}
        f_4(0, x)     & = & 1 \\
        f_4(y + 1, x) & = & P_2^3(f_4(y, f_4(y, x), x)) \\
      \end{eqnarray*}
  \end{enumerate}

\end{enumerate}

\end{document}

