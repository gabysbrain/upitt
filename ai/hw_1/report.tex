\documentclass[10pt]{article}

\usepackage{verbatim}
\usepackage{fullpage}

%page formatting
\begin{document}

{\bf Homework 1} \hfill {\raggedleft Thomas Torsney-Weir}

\section{Implementation}
We are testing the IDS, A* with Manhattan distance, and A* with Euclidean
distance search algorithms.

\subsection{Running}
The program to solve a puzzle is called {\tt puzzle}.  It takes two arguments.
The first argument selects the search algorithm.  It can be {\tt i}, {\tt m}, 
or {\tt e} which indicates IDS, A* with Manhattan distance, and A* with
Euclidean distance respectively.  The second argument is the puzzle.  This
is a string of nine numbers which determine the arangement of the puzzle.  The
goal is always \verb@123456780@.

The application prints five numbers to stdout.  These are states visited, 
states expanded, solution length, max search depth, and max fringe size in
order from left to right.

\subsection{Validity}
The following are outputs showing the internal computions for various 
pieces of the program.  These can all be turned on by recompiling with
certain flags \verb1#define1-ed.

\subsubsection{Successor}
Define \verb1SUCC_DEBUG1 to activate this output.  {\tt orig} is the input
puzzle and each {\tt next} line are the successors.

\begin{verbatim}
succ: orig: [4 1 2 7 6 3 0 5 8]
      next: [4 1 2 0 6 3 7 5 8]
      next: [4 1 2 7 6 3 5 0 8]
succ: orig: [4 1 2 7 6 3 0 5 8]
      next: [4 1 2 0 6 3 7 5 8]
      next: [4 1 2 7 6 3 5 0 8]
succ: orig: [4 1 2 0 6 3 7 5 8]
      next: [4 1 2 7 6 3 0 5 8]
      next: [4 1 2 6 0 3 7 5 8]
succ: orig: [4 1 2 7 6 3 5 0 8]
      next: [4 1 2 7 0 3 5 6 8]
      next: [4 1 2 7 6 3 0 5 8]
      next: [4 1 2 7 6 3 5 8 0]
\end{verbatim}

\subsubsection{Goal Test}
Define \verb1GOAL_DEBUG1 to activate this output.  {\tt inpt} is the input
puzzle, {\tt goal} is the goal state, and {\tt bool} will be 1 if the puzzle
is equal to the goal.

\begin{verbatim}
goal: inpt: [1 2 3 4 5 0 7 8 6]
      goal: [1 2 3 4 5 6 7 8 0]
      bool: 0
goal: inpt: [1 2 0 4 5 3 7 8 6]
      goal: [1 2 3 4 5 6 7 8 0]
      bool: 0
goal: inpt: [1 2 3 4 5 6 7 8 0]
      goal: [1 2 3 4 5 6 7 8 0]
      bool: 1
\end{verbatim}

\subsubsection{Manhattan Heuristic}
Define \verb1MAN_DEBUG1 to activate this output.  {\tt puzz} is the input
puzzle, {\tt goal} is the goal state, and {\tt dist} is the manhattan distance.

\begin{verbatim}
mand: puzz: [1 2 3 4 5 6 0 8 7]
      goal: [1 2 3 4 5 6 7 8 0]
      dist: 2
mand: puzz: [1 2 3 0 5 6 4 8 7]
      goal: [1 2 3 4 5 6 7 8 0]
      dist: 3
mand: puzz: [1 2 3 4 5 6 8 0 7]
      goal: [1 2 3 4 5 6 7 8 0]
      dist: 3
mand: puzz: [1 0 3 5 2 6 4 8 7]
      goal: [1 2 3 4 5 6 7 8 0]
      dist: 5
\end{verbatim}

\subsubsection{Euclidean Heuristic}
Define \verb1EUC_DEBUG1 to activate this output.  {\tt inpt} is the input
puzzle, {\tt goal} is the goal state, and {\tt dist} is the euclidean distance.

\begin{verbatim}
eucd: puzz: [1 2 3 4 5 6 0 8 7]
      goal: [1 2 3 4 5 6 7 8 0]
      dist: 4
eucd: puzz: [1 2 3 0 5 6 4 8 7]
      goal: [1 2 3 4 5 6 7 8 0]
      dist: 5.23607
eucd: puzz: [1 2 3 4 0 6 8 5 7]
      goal: [1 2 3 4 5 6 7 8 0]
      dist: 5.41421
eucd: puzz: [1 2 3 4 5 6 8 7 0]
      goal: [1 2 3 4 5 6 7 8 0]
      dist: 2
\end{verbatim}

\subsection{Experiment}
To test, we ran 100 puzzles for each solution length from 1 to 25.  Each
puzzle was run using IDS, A* with the Manhattan distance heuristic, and
A* with the Euclidean distance heuristic.  The results are below.

\section{Results}
\begin{tiny}
\begin{tabular}{*{10}{|c}|}
  & \multicolumn{3}{|c|}{IDS} & \multicolumn{3}{|c|}{A* Manhattan} & 
    \multicolumn{3}{|c|}{A* Euclidean} \\
d & expanded & max fringe & branch factor & 
    expanded & max fringe & branch factor & 
    expanded & max fringe & branch factor\\
0 & 0 & 0 & None & 0 & 0 & None & 0 & 0 & None\\
1 & 3 & 3 & 3.999 & 3 & 3 & 1.999 & 3 & 3 & 1.999\\
2 & 15 & 9 & 3.405 & 6 & 4 & 1.302 & 6 & 4 & 1.302\\
3 & 47 & 22 & 3.029 & 9 & 5 & 1.150 & 9 & 5 & 1.150\\
4 & 119 & 52 & 2.841 & 12 & 6 & 1.091 & 12 & 6 & 1.091\\
5 & 410 & 171 & 2.898 & 15 & 7 & 1.061 & 17 & 7 & 1.114\\
6 & 1203 & 494 & 2.934 & 19 & 8 & 1.044 & 37 & 12 & 1.248\\
7 & 3099 & 1263 & 2.865 & 24 & 10 & 1.062 & 52 & 17 & 1.252\\
8 & 8074 & 3283 & 2.819 & 24 & 10 & 1.026 & 65 & 20 & 1.242\\
9 & 23643 & 9600 & 2.831 & 29 & 11 & 1.039 & 37 & 13 & 1.087\\
10 & 56218 & 22817 & 2.797 & 47 & 16 & 1.094 & 60 & 20 & 1.139\\
11 & 17927 & 72745 & 2.812 & 51 & 17 & 1.088 & 67 & 21 & 1.125\\
12 & 53863 & 21854 & 2.844 & 65 & 20 & 1.102 & 80 & 25 & 1.130\\
13 & 13498 & 54766 & 2.811 & 84 & 25 & 1.118 & 87 & 27 & 1.118\\
14 & 37404 & 15175 & 2.808 & 118 & 33 & 1.143 & 116 & 36 & 1.134\\
15 & 10135 & 41122 & 2.797 & 126 & 35 & 1.130 & 109 & 31 & 1.115\\
16 & 23322 & 94625 & 2.755 & 117 & 33 & 1.108 & 112 & 32 & 1.103\\
17 & 77220 & 31329 & 2.786 & 121 & 34 & 1.100 & 115 & 33 & 1.094\\
18 & 15366 & 62343 & 2.724 & 129 & 35 & 1.095 & 115 & 33 & 1.083\\
19 & - & - & - & 143 & 39 & 1.094 & 109 & 33 & 1.067\\
20 & - & - & - & 133 & 36 & 1.079 & 116 & 35 & 1.064\\
21 & - & - & - & 132 & 35 & 1.071 & 125 & 37 & 1.064\\
22 & - & - & - & 215 & 54 & 1.102 & 125 & 38 & 1.058\\
23 & - & - & - & 212 & 54 & 1.092 & 126 & 38 & 1.052\\
24 & - & - & - & 195 & 50 & 1.080 & 130 & 39 & 1.050\\
25 & - & - & - & 171 & 44 & 1.066 & 130 & 40 & 1.044\\
\end{tabular}
\end{tiny}

\section{Analysis}
IDS performs very poorly on this problem.  It simply keeps too many states
in memory.  It could not solve puzzles with solution lengths greater than 18.
The program would run out of memory before it solved them.  This can be seen
in the maximum fringe in the table.  This is the maximum number of nodes kept
in memory at each search step.  The fringe is also expanding exponentially
for the solution depth.  Also note the branching factor.  It never drops below
2.5 for any solution size.  
The strength of IDS is that it is simpler to 
implement since we do not have to come up with a heuristic function.  However, 
it expands far too many states before finding a solution and this prevents it 
from solving longer puzzles.

A* performs far better than IDS.  Using either heuristic it performed far 
better then IDS.  In fact, it found solutions where IDS failed.  It also kept
fewer nodes in memory and expanded fewer states.  Furthermore, the branching 
factor for either heuristic is very close to one.  This means that A* is
looking at about one node at each state.  It is making few mistakes finding
the solution.  This is also backed up by the number of nodes expanded and
the max fringe size.  It is increasing almost linearly with the size of the 
solution.

The performance of these two heuristics is difficult to compare.  The number 
of expanded nodes, max fringe size, and branching factor are very similar.  
This makes sense as these two heuristics are almost the same algorithm.  
Manhattan distance returns the sum of the horizonal and vertical error distance
while Euclidean distance is just the hypotenuse of that triangle.  You use
the exact same 2 measurements --- vertical error and horizontal error --- for
both heuristics.  Either of these performs very well on puzzles of any size.

\end{document}
